%%%%%%%%%%%%%%%%%%%%%%%%%%%%%%
% Ioannis Petrousov
% petrousov@gmail.com
% 2026
%
% Initial compile:
%     1. xelatex
%     2. bibtex
%     3. xelatex
%     4. xelatex
%
% Further compile:
%
% A: If you change the bibliography file
%     1. biblatex
%     2. xelatex
%     3. xelatex
%
% B: If you DONT change the bibliography file
%     1. xelatex
%     2. xelatex
%%%%%%%%%%%%%%%%%%%%%%%%%%%%%%

% Use the following documentclass declaration if you plan to make a hardcopy
% The twoside option shift text for making a book print
%\documentclass[12pt,twoside]{report}

% Use the following documentclass for PDF and to view on computer only
\documentclass[11pt,a4paper]{report}%{book}

% Language settings
\usepackage[utf8]{inputenc}
\usepackage[greek,english]{babel}

% TEXT FORMATTING
% set spacing between lines (line spacing)
\usepackage{setspace}
\setstretch{1.5}

% package to customize chapters, sections and subsections style
\usepackage{titlesec}
% Section: Single number (1, 2, 3)
\renewcommand{\thesection}{\arabic{section}}
% Subsection: Section dot Subsection (1.1, 1.2)
\renewcommand{\thesubsection}{\thesection.\arabic{subsection}}

% chapter title appearance format
\titleformat{\chapter}[display]
{\bfseries\huge}{\chaptertitlename\space\thechapter}{16pt}{}
% https://www.sharelatex.com/learn/Sections_and_chapters
\titlespacing{\chapter}{0pc}{1.5ex plus .1ex minus .2ex}{5pc}

% section title appearance format
\titleformat{\section}
{\bfseries\large}{\thesection}{14pt}{}


% subsection title appearance format
\titleformat{\subsection}
{\bfseries\normalsize}{\thesubsection}{12pt}{}

% set margins
\usepackage{geometry}
\geometry{left=3cm, right=2cm, top=2.5cm, bottom=2.5cm}
% text justification is fully-justified by default

% Numbered lists package
\usepackage{enumitem}

% xelatex wants fontspec
\usepackage{fontspec}
\usepackage{xltxtra}
% \setmainfont[Mapping=tex-text]{Latin Modern Roman}
\setmainfont[Mapping=tex-text]{XITS}
% package allows to place figures
\usepackage{graphicx}
% put images in images path
\graphicspath{{images/}}
\usepackage{setspace}

% Caption customization
% use this package to set appearance for captions
\usepackage{caption}
% caption size for figures 10pt
\captionsetup[figure]{font=footnotesize,labelfont=footnotesize}
% caption size for tables 10pt and underlined
\usepackage[normalem]{ulem} % Package for underlining
\DeclareCaptionLabelFormat{label_format}{\uline{#1~#2}} % underline label
\DeclareCaptionTextFormat{text_format}{\uline{#1}} % underline text
\DeclareCaptionLabelSeparator{separator_format}{\uline{:~}} % underline separator
%\captionsetup[table]{font=normalsize,labelfont=normalsize,labelformat=label_format,textformat=text_format,labelseparator=separator_format}

% use this package to define custom colors
\usepackage{xcolor}

% create colors
\colorlet{punct}{red!60!black}
\definecolor{background}{HTML}{EEEEEE}
\definecolor{delim}{RGB}{20,105,176}
\colorlet{numb}{magenta!60!black}

% use this package for algorithms
\usepackage{algorithm2e}

% use math package to create equations
\usepackage{amsmath}

% create command for blank page
\usepackage{afterpage}
\newcommand\blankpage{%
    \null
    \thispagestyle{empty}%
    \addtocounter{page}{-1}%
    \newpage}

% add clickable hyperlinks
\usepackage{hyperref}
\hypersetup{
    colorlinks,
    citecolor=black,
    filecolor=black,
    linkcolor=black,
    urlcolor=black
}

% use fancy header and footer
\usepackage{fancyhdr}
\usepackage{blindtext} % to quickly get a full document

% Turn on the style
\pagestyle{fancy}

% Clear the header and footer
\fancyhf{}

% Set the right side of the footer to be the page number
\fancyfoot[R]{\thepage}

% set page number appearance to bottom right
\fancypagestyle{plain}{%
    \renewcommand{\headrulewidth}{0pt}
    \fancyhf{}
    \fancyfoot[R]{\thepage}%
}

\newcommand{\HRule}{\rule{\linewidth}{0.5mm}}

% change the word 'Algorithm' in caption to Algorithm
\renewcommand*{\algorithmcfname}{Algorithm}

% Change "List of Algorithms" to "List of Algorithms"
\renewcommand\listalgorithmcfname{List of Algorithms}

% package to create and customize appendixes (Appendices)
% put appendix title to table of contents (toc)
% Puts a title (e.g., ‘Appendices’) into the document at the point where the appendices environment is begun (page)
\usepackage[toc,page]{appendix}
% Use fancy tables
\usepackage{booktabs}
% change appendix name to Appendices in table of content
\renewcommand{\appendixtocname}{Appendices}
% change appendix name to Appendices in page
\renewcommand{\appendixpagename}{Appendices}

% Allows item and enum customization
\usepackage{enumitem}


\begin{document}
\begin{titlepage}
\begin{center}

\begin{figure}[h]
\centering

\includegraphics[width=0.4\textwidth]{images/program_logo.png}
\caption*{D.P.M.S.\\ Artificial Intelligence} % Unnumbered text for the specific program
\end{figure}

% Horizontal row of logos
\begin{center}
    % Define a fixed, consistent height for both images (e.g., 2cm)
    % Use minipages to guarantee they are placed side-by-side

    % Left Logo
    \begin{minipage}{0.48\textwidth} % Takes up almost half the line
        \centering
        \includegraphics[height=2cm]{images/unipi_logo.png}
    \end{minipage}%
    \hfill % This command pushes the two minipages to the far edges of the center environment
    \begin{minipage}{0.48\textwidth} % Takes up almost half the line
        \centering
        \includegraphics[height=2cm]{images/demokritos_logo.png}
    \end{minipage}
\end{center}

\begin{center}
% leave 2 cm from above text
\vspace{2cm}

\HRule \\[0.4cm]
{\huge Report on paper: Learning to Bid Without Knowing your Value\\}
\HRule \\[0.4cm]
\end{center}

Course: Algorithmic something

% put this on the bottom
\vfill
\begin{doublespacing}

{\LARGE
Ioannis Petrousov, Faidon\\}
\vfill
{\Large \ifcase\month\or January\or February\or March\or April\or May\or June\or July\or August\or September\or October\or November\or December\fi, \number\year}
\end{doublespacing}
\end{center}
\end{titlepage}

% Insert copyright
\chapter*{Declaration of Intellectual Property Rights}

\noindent{Copyright (C) Ioannis Petrousov, Faidon}

% insert table of contents
\newpage\tableofcontents

% insert list of figures
\listoffigures
\clearpage


% insert list of tables
\listoftables
\clearpage

% leave blank page before main part
\blankpage


% CONTENT



\section{Εισαγωγή}

\subsection{Περιγραφή του προβλήματος}

Στη παραδοσιακή θεωρία παιγνίων, οι συμμετέχοντες γνωρίζουν την αξία των αγαθών προς πώληση.
Έχοντας αυτή τη γνώση, μπορούν να μεγιστοποιήσουν τα κέρδη τους πλειοδοτώντας όσο γίνεται πιο κοντά
στην πραγματική αξία του αγαθού. Αυτή η παραδοχή λειτουργεί σε περιβάλλοντα όπου ο αριθμός
συναλλαγών και προϊόντων είναι σχετικά μικρός. Ωστόσο, παραβιάζεται σε περιβάλλοντα ψηφιακής
οικονομίας στα οποία συμμετέχει ένας μεγάλος αριθμός πλειοδοτών, οι οποίοι ανταγωνίζονται
μεταξύ τους χωρίς γνώση της υποκειμενικής αξίας ($v_t$) του αγαθού. Αυτό δυσκολεύει τον προσδιορισμό της
ιδανικής προσφοράς και δημιουργίας βέλτιστης στρατηγικής που μεγιστοποιεί το κέρδος. Παραδείγματα τέτοιων
περιβαλλόντων αποτελούν οι δημοπρασίες στο eBay ή οι διαδικτυακές διαφημιστικές δημοπρασίες.

%---%

Το άρθρο πραγματεύεται την πρόκληση της διαδικτυακής μάθησης (online learning) σε σύνθετα περιβάλλοντα
δημοπρασιών όπου η αξία του πλειοδότη ($v_t$) είναι άγνωστη, μεταβάλλεται αυθαίρετα και γίνεται ορατή μόνο σε περίπτωση
που ο πλειοδότης κερδίσει μία δημοπρασία (win-only feedback). Η απόκτηση αυτής της γνώσης (της αξίας) συνεπάγεται
υψηλότερο κόστος προσφοράς από τον πλειοδότη. Έτσι δημιουργείται μια εγγενή αντιστάθμιση μεταξύ της μάθησης της αξίας
και του κόστους συμμετοχής.

%---%

Στόχος κάθε πλειοδότη είναι η ελαχιστοποίηση της μεταμέλειας (regret) σε σχέση με την καλύτερη προσφορά ($b^{*}$) που
θα μπορούσε να είχε επιλεγεί εκ των υστέρων, αν όλες οι παράμετροι ήταν γνωστές.

%---%

\subsection{Συναφή έρευνα}

Η παραπάνω πρόκληση μπορεί να προσεγγιστεί με το μοντέλο των Multi-Armed Bandit (MAB), στο οποίο κάθε πιθανή
προσφορά αντιμετωπίζεται ως ένας ανεξάρτητος "βραχίονας" μηχανής κουλοχέρη. Σε αυτή την περίπτωση, ο βαθμός μεταμέλειας
κλιμακώνεται ανάλογα με το πλήθος των προσφορών: $O(\sqrt{T|B|})$. Αυτή η τεχνική θεωρείται αργή για τα δεδομένα των
σύγχρονων δημοπρασιών, καθώς δεν εκμεταλλεύεται τη μαθηματική δομή του προβλήματος και τη διαθέσιμη μερική
ανατροφοδότηση.

Έρευνες που προσπαθούν να αξιοποιήσουν την παραπάνω πληροφορία περιλαμβάνουν αλγορίθμους που βασίζονται
αποκλειστικά σε δημοπρασίες δεύτερης τιμής (second-price auctions). Το βασικό μειονέκτημα αυτών των προσεγγίσεων
είναι ότι είναι εξειδικευμένες και αδυνατούν να προσαρμοστούν σε πιο γενικευμένες ή σύνθετες περιπτώσεις.


Τέλος, επισημαίνεται ότι στην περίπτωση που ο πλειοδότης γνωρίζει πλήρως την αξία του αγαθού ($v_t$), οι ρυθμοί
σύγκλισης είναι πολύ ταχύτεροι. Ωστόσο, αυτή η γνώση σπάνια προϋπάρχει στην πράξη στα δυναμικά περιβάλλοντα
της ψηφιακής οικονομίας.

%---%

\subsection{Συνεισφορές της έρευνας}

Η κύρια συνεισφορά του άρθρου είναι η εισαγωγή ενός νέου πλαισίου μάθησης με \textbf{πληροφόρηση βάσει αποτελέσματος (outcome-based feedback)}.
Το μοντέλο αυτό γενικεύει προηγούμενες προσεγγίσεις και ακολουθεί τα εξής βήματα σε κάθε γύρο $t$:

\begin{itemize}[itemsep=-2pt, parsep=0pt]
    \item Ο πλειοδότης επιλέγει μια προσφορά $b$ από ένα σύνολο $B$.
    \item Ο αντίπαλος (adversary) επιλέγει μια συνάρτηση κατανομής $x_t$​ (π.χ. πιθανότητα κλικ) και μια συνάρτηση ανταμοιβής $r_t$
    \item Προκύπτει ένα τυχαίο αποτέλεσμα $o_t$ (π.χ. "κλικ" ή "όχι κλικ") βάσει της κατανομής $x_t(b)$.

    \item Ο πλειοδότης παρατηρεί τη συνάρτηση $x_t$​ και τη συνάρτηση ανταμοιβής $r_t(\cdot,o_t)$ για συγκεκριμένο αποτέλεσμα $o_t$​ που προέκυψε.
\end{itemize}

Αυτή η προσέγγιση επιτρέπει την εφαρμογή του αλγορίθμου \textbf{WIN-EXP}, ο οποίος επιτυγχάνει ρυθμούς μεταμέλειας που εξαρτώνται λογαριθμικά από τον αριθμό των δράσεων και γραμμικά από τον αριθμό των πιθανών αποτελεσμάτων $O$:
$O(\sqrt{T|O|\log(|B|)})$

%---%

\clearpage
\section{The outcome-based feedback framework}

\clearpage
\section{WIN-EXP Αλγόριθμος}

\clearpage
\section{Επεκτάσεις του αλγορίθμου}

\clearpage
\section{Πειράματα και Σύναψη}


\end{document}
