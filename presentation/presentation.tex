% About
% Dark Beamer theme
% https://github.com/gpetrousov/dark-beamer-theme
%
% Author
% Original author: Blottiere Paul
% Additions: Ioannis Petrousov
%
% Compile
% just run xelatex

% aspect ratio 16:9
\documentclass[12pt,t,aspectratio=169,xcolor=table]{beamer}

% aspect ratio 4:3
% WARNING
% If you decide to use this ration, make sure you adjust the
% position of the date so it sit's exactly on the footer (blue line).
% In general, avoid using this ratio, we 're not in the 90's anymore.
%\documentclass[12pt,t]{beamer}

% Set theme
\usetheme{Madrid}

% usepackage
% \usepackage{theme/dbt} % our main theme
\usepackage{algorithm2e}
\usepackage{xltxtra}

% Font related work - adapted to Greek
\usepackage{fontspec}
\usepackage[greek,english]{babel}
\usefonttheme{serif}
\setmainfont{GFS Didot}
\setsansfont{GFS Didot}

\usepackage{listings}
\usepackage{multirow} % allows merging cells in tables
\usepackage{hyperref}
\usepackage{caption}

% set color of table caption
\captionsetup[table]{labelfont={color=yellow}}

% set bibliography colors
\setbeamercolor{bibliography item}{fg=white,bg=white}
\setbeamercolor*{bibliography entry title}{fg=white,bg=white}
\setbeamercolor*{bibliography entry author}{fg=keywords,bg=keywords}

% put images in images path
% \graphicspath{{images/}}

\title[Report on paper:Learning to Bid Without Knowing your Value]{Report on paper:Learning to Bid Without Knowing your Value}
\date{\ifcase\month\or January\or February\or March\or April\or May\or June\or July\or August\or September\or October\or November\or December\fi, \number\year}
% \subtitle{
% Πανεπιστήμιο Δυτικής Μακεδονίας\\
% Τμήμα Μηχανικών Πληροφορικής και Τηλεπικοινωνιών
% }

\author[Ioannis Petrousov, Faidon]{}

% Affiliations
% \institute {
% Εργαστήριο Ψηφιακών Συστημάτων και Αρχιτεκτονικής Υπολογιστών\\
% http://arch.icte.uowm.gr
% }

\begin{document} {
    \setbeamertemplate{footline}{}
    \frame {
        \titlepage
    }
}

\begin{frame}{Πίνακας περιεχομένων}
    \tableofcontents
\end{frame}


\section{Εισαγωγή}
\begin{frame}
    \frametitle{Εισαγωγή}
    \begin{itemize}
        \item \textbf{Παραδοσιακή Παραδοχή:} Οι συμμετέχοντες έχουν σαφή γνώση της αξίας (value) των αγαθών.
        \item \textbf{Ψηφιακή Οικονομία:} Η παραδοχή παραβιάζεται σε περιβάλλοντα όπως οι διαφημίσεις αναζήτησης ή πλειστηριασμοί eBay.
        \item \textbf{Το Πρόβλημα:} Η αξία ενός "κλικ" ή ενός αγαθού είναι συχνά άγνωστη πριν τη δημοπρασία.
        \item \textbf{Δυναμική Μάθησης:} Οι πλειοδότες πρέπει να μαθαίνουν «πράττοντας» (learning-by-doing) μέσω επαναλαμβανόμενων συμμετοχών.
    \end{itemize}
\end{frame}

\subsection{Συναφή έρευνα}
\begin{frame}{Συναφή Έρευνα και Περιορισμοί}
    \begin{itemize}
        \item \textbf{Multi-Armed Bandits (MAB):} Κάθε προσφορά αντιμετωπίζεται ως ανεξάρτητος «βραχίονας».
        \item \textbf{Ρυθμός Μεταμέλειας (Regret):} Οι κλασικοί αλγόριθμοι MAB έχουν μεταμέλεια $O(\sqrt{T|B|})$, η οποία αυξάνεται γραμμικά με το πλήθος των προσφορών $|B|$.
        \item \textbf{Εξειδικευμένες Λύσεις:} Προηγούμενες εργασίες εστιάζουν κυρίως σε δημοπρασίες δεύτερης τιμής, στερούμενες γενίκευσης.
        \item \textbf{Έλλειμμα Δομής:} Οι τυπικοί αλγόριθμοι δεν εκμεταλλεύονται τη δομή της χρησιμότητας και της μερικής ανατροφοδότησης των δημοπρασιών.
    \end{itemize}
\end{frame}


\subsection{Συνεισφορές της έρευνας}
\begin{frame}{Κύριες Συνεισφορές του Άρθρου}
    \begin{itemize}
        \item \textbf{Outcome-Based Feedback:} Εισαγωγή ενός νέου πλαισίου μάθησης που γενικεύει τα feedback graphs.
        \item \textbf{Αλγόριθμος WIN-EXP:} Μια παραλλαγή του EXP3 με αμερόληπτες εκτιμήσεις χαμηλής διασποράς.
        \item \textbf{Εκθετική Επιτάχυνση:} Ρυθμός μεταμέλειας $O(\sqrt{T|O|\log(|B|)})$, με λογαριθμική εξάρτηση από τις προσφορές.
        \item \textbf{Καθολική Εφαρμογή:} Κατάλληλο για δημοπρασίες πρώτης και δεύτερης τιμής, sponsored search και multi-item auctions.
        \item \textbf{Ανθεκτικότητα:} Αποδεδειγμένη ανθεκτικότητα σε θορυβώδη δεδομένα και σφάλματα εκτίμησης.
    \end{itemize}
\end{frame}

%--------%


\section{Ο αλγόριθμος WIN-EXP}
\begin{frame}{Ο Αλγόριθμος WIN-EXP}
    \begin{itemize}
        \item \textbf{Προέλευση:} Αποτελεί μια παραλλαγή του αλγορίθμου EXP3.
        \item \textbf{Λειτουργία:} Διατηρεί μια κατανομή πιθανότητας $\pi_t$ πάνω στο σύνολο των δυνατών προσφορών $B$.
        \item \textbf{Καινοτομία:} Ενσωματώνει την ανατροφοδότηση της συνάρτησης κατανομής $x_t$ για την κατασκευή εκτιμήσεων.
        \item \textbf{Διαδικασία:} 
            \begin{enumerate}
                \item Επιλογή προσφοράς $b_t$ από την κατανομή $\pi_t$.
                \item Παρατήρηση της πιθανότητας νίκης $x_t$ και, σε περίπτωση νίκης, της ανταμοιβής $r_t$.
                \item Υπολογισμός εκτίμησης χρησιμότητας και ενημέρωση των βαρών $\pi_t$.
            \end{enumerate}
    \end{itemize}
\end{frame}

\subsection{H Αμερόληπτη Εκτίμηση Χρησιμότητας}
\begin{frame}{H Αμερόληπτη Εκτίμηση Χρησιμότητας}
    \begin{itemize}
        \item \textbf{Στόχος:} Κατασκευή μιας εκτίμησης με χαμηλή διασπορά (variance) που αξιοποιεί τη δομή της δημοπρασίας.
        \item \textbf{Ορισμός (για win-only feedback):}
            \begin{itemize}
                \item Αν κερδηθεί η ανταμοιβή: $\tilde{u}_t(b) = \frac{(r_t(b)-1)Pr[A_t|b_t=b]}{Pr[A_t]}$
                \item Αν δεν κερδηθεί η ανταμοιβή: $\tilde{u}_t(b) = -\frac{Pr[\neg A_t|b_t=b]}{Pr[\neg A_t]}$
            \end{itemize}
        \item \textbf{Ιδιότητα (Λήμμα 3.1):} Η τυχαία μεταβλητή $\tilde{u}_t(b)$ είναι μια αμερόληπτη εκτίμηση της πραγματικής αναμενόμενης χρησιμότητας $u_t(b)$, μετατοπισμένη κατά $-1$ ($\mathbb{E}[\tilde{u}_t(b)] = u_t(b) - 1$).
    \end{itemize}
\end{frame}


\subsection{Μεταμέλεια}
\begin{frame}
    \begin{center}
        \Huge Μεταμέλεια
    \end{center}
\end{frame}

%--------%


\section{Lipschitz}
\begin{frame}
    \begin{center}
        \Huge Lipschitz
    \end{center}
\end{frame}

\subsection{sub 1}
\begin{frame}
    \begin{center}
        \Huge sub 1
    \end{center}
\end{frame}


\subsection{Sub 2}
\begin{frame}
    \begin{center}
        \Huge Sub 2
    \end{center}
\end{frame}

%--------%

\section{Πειράματα και Σύνοψη}
\begin{frame}
    \begin{center}
        \Huge Πειράματα
    \end{center}
\end{frame}

\subsection{Σύνοψη}
\begin{frame}
    \begin{center}
        \Huge Σύνοψη
    \end{center}
\end{frame}

%--------%

\section{Τέλος}
\begin{frame}[plain,c]
\begin{center}
\Huge Τέλος
% \color{title}\small\vfill contact@petrousoft.com\\
\end{center}
\end{frame}




\end{document}
