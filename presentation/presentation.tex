%%%%%%%%%%%%%%%%%%%%%%%%%%%%%%
% Ioannis Petrousov
% Report Presentation
% 2026
%%%%%%%%%%%%%%%%%%%%%%%%%%%%%%

\documentclass[11pt,t,aspectratio=169,xcolor=table]{beamer}

% Set theme
\usetheme{Madrid}

% Packages
\usepackage{algorithm2e}
\usepackage{xltxtra}
\usepackage{fontspec}
\usepackage[greek,english]{babel}
\usepackage{listings}
\usepackage{multirow}
\usepackage{hyperref}
\usepackage{caption}
\usepackage{amsmath}
\usepackage{qrcode}
\usepackage{graphicx}
\usepackage{hyperref}
\usepackage{caption}

% Font settings
\usefonttheme{serif}
\setmainfont{GFS Didot}
\setsansfont{GFS Didot}

% Bibliography colors (standard Beamer)
\setbeamercolor{bibliography item}{fg=black}
\setbeamercolor{bibliography entry author}{fg=black}
\setbeamercolor{bibliography entry title}{fg=black}
\setbeamercolor{bibliography entry location}{fg=black}
\setbeamercolor{bibliography entry note}{fg=black}

\title[Report: Learning to Bid Without Knowing your Value]{Report on paper: Learning to Bid Without Knowing your Value}
\author[Ioannis Petrousov]{Ioannis Petrousov}
\date{\ifcase\month\or January\or February\or March\or April\or May\or June\or July\or August\or September\or October\or November\or December\fi, \number\year}

\begin{document}

{
    \setbeamertemplate{footline}{}
    \frame{\titlepage}
}

\begin{frame}{Πίνακας περιεχομένων}
    \tableofcontents
\end{frame}

\section{Εισαγωγή}
\begin{frame}{Εισαγωγή}
    \begin{itemize}
        \item \textbf{Παραδοσιακή Παραδοχή:} Οι συμμετέχοντες έχουν σαφή γνώση της αξίας (value) των αγαθών.
        \item \textbf{Ψηφιακή Οικονομία:} Η παραδοχή παραβιάζεται σε περιβάλλοντα όπως οι διαφημίσεις αναζήτησης ή πλειστηριασμοί eBay.
        \item \textbf{Το Πρόβλημα:} Η αξία ενός "κλικ" ή ενός αγαθού είναι συχνά άγνωστη πριν τη δημοπρασία.
        \item \textbf{Δυναμική Μάθησης:} Οι πλειοδότες πρέπει να μαθαίνουν "πράττοντας" (learning-by-doing) μέσω επαναλαμβανόμενων συμμετοχών.
    \end{itemize}
\end{frame}

\subsection{Συναφή έρευνα}
\begin{frame}{Συναφή Έρευνα και Περιορισμοί}
    \begin{itemize}
        \item \textbf{Multi-Armed Bandits (MAB):} Κάθε προσφορά αντιμετωπίζεται ως ανεξάρτητος "βραχίονας".
        \item \textbf{Ρυθμός Μεταμέλειας (Regret):} Οι κλασικοί αλγόριθμοι MAB έχουν μεταμέλεια $O(\sqrt{T|B|})$, η οποία αυξάνεται γραμμικά με το πλήθος των προσφορών $|B|$.
        \item \textbf{Εξειδικευμένες Λύσεις:} Προηγούμενες εργασίες εστιάζουν κυρίως σε δημοπρασίες δεύτερης τιμής, στερούμενες γενίκευσης.
        \item \textbf{Έλλειμμα Δομής:} Οι τυπικοί αλγόριθμοι δεν εκμεταλλεύονται τη δομή της χρησιμότητας και της μερικής ανατροφοδότησης των δημοπρασιών.
    \end{itemize}
    
\end{frame}

\subsection{Συνεισφορές της έρευνας}
\begin{frame}{Κύριες Συνεισφορές του Άρθρου}
    \begin{itemize}
        \item \textbf{Outcome-Based Feedback:} Εισαγωγή ενός νέου πλαισίου μάθησης που γενικεύει τα feedback graphs.
        \item \textbf{Αλγόριθμος WIN-EXP:} Μια παραλλαγή του EXP3 με αμερόληπτες εκτιμήσεις χαμηλής διασποράς.
        \item \textbf{Εκθετική Επιτάχυνση:} Ρυθμός μεταμέλειας $O(\sqrt{T|O|\log(|B|)})$, με λογαριθμική εξάρτηση από τις προσφορές.
        \item \textbf{Καθολική Εφαρμογή:} Κατάλληλο για δημοπρασίες πρώτης και δεύτερης τιμής, sponsored search και multi-item auctions.
    \end{itemize}
\end{frame}

\section{Ο αλγόριθμος WIN-EXP}
\begin{frame}{Ο Αλγόριθμος WIN-EXP}
    \begin{itemize}
        \item \textbf{Προέλευση:} Αποτελεί μια παραλλαγή του αλγορίθμου EXP3.
        \item \textbf{Λειτουργία:} Διατηρεί μια κατανομή πιθανότητας $\pi_t$ πάνω στο σύνολο των δυνατών προσφορών $B$.
        \item \textbf{Καινοτομία:} Ενσωματώνει την ανατροφοδότηση της συνάρτησης κατανομής $x_t$ για την κατασκευή εκτιμήσεων.
        \item \textbf{Διαδικασία:} 
            \begin{enumerate}
                \item Επιλογή προσφοράς $b_t$ από την κατανομή $\pi_t$.
                \item Παρατήρηση της πιθανότητας νίκης $x_t$ και, σε περίπτωση νίκης, της ανταμοιβής $r_t$.
                \item Ενημέρωση των βαρών: $\pi_{t+1}(b) \propto \pi_t(b) e^{\eta \tilde{u}_t(b)}$.
            \end{enumerate}
    \end{itemize}
\end{frame}

\subsection{H Αμερόληπτη Εκτίμηση Χρησιμότητας}
\begin{frame}{H Αμερόληπτη Εκτίμηση Χρησιμότητας}
    \textbf{Ορισμός (για win-only feedback):}
    \begin{block}{Εκτίμηση Χρησιμότητας $\tilde{u}_t(b)$}
        \begin{itemize}
            \item Αν κερδηθεί η δημοπρασία ($A_t$): $\tilde{u}_t(b) = \frac{(r_t(b)-1)Pr[A_t|b_t=b]}{Pr[A_t]}$
            \item Αν δεν κερδηθεί η δημοπρασία ($\neg A_t$): $\tilde{u}_t(b) = -\frac{Pr[\neg A_t|b_t=b]}{Pr[\neg A_t]}$
        \end{itemize}
    \end{block}
    \begin{itemize}
        \item \textbf{Ιδιότητα (Λήμμα 3.1):} $\mathbb{E}[\tilde{u}_t(b)] = u_t(b) - 1$.
        \item Επιτρέπει τη μάθηση για \textbf{όλες} τις προσφορές $b$ από ένα μόνο αποτέλεσμα (Importance Sampling).
    \end{itemize}
\end{frame}

\section{Διακριτοποίηση και Lipschitz Συνέχεια}
\begin{frame}{Διακριτοποίηση και Lipschitz Συνέχεια}
    \textbf{Πρόβλημα:} Ο χώρος προσφορών είναι συνεχής, αλλά ο αλγόριθμος απαιτεί διακριτά σημεία.
    
    \begin{itemize}
        \item \textbf{Διακριτοποίηση (Binning):} Χωρίζουμε το συνεχές διάστημα σε $K$ επιμέρους διαστήματα (\textbf{bins}) \cite{medium_discretization_2020}.
        \item \textbf{$\Delta$-Lipschitz Συνέχεια:} Εγγυάται ότι η αναμενόμενη χρησιμότητα είναι "ομαλή" (smooth).
        \item \textbf{Φράγμα Μεταμέλειας (Theorem 5.4):} Στον συνεχή χώρο είναι $O(T^{2/3}\sqrt{\log T})$.
        \item \textbf{Trade-off:} Εξισορρόπηση σφάλματος εκτίμησης (estimation error) και σφάλματος πλέγματος (discretization error).
    \end{itemize}
    
\end{frame}

\subsection{Πρακτική Εφαρμογή}
\begin{frame}{Πρακτική Εφαρμογή και Bid Simulators}
    \begin{itemize}
        \item \textbf{Η Φύση της $x_t(b)$:} Αντιστοιχεί στην \textbf{CDF} των προσφορών των αντιπάλων \cite{wiki_cdf}.
        \item \textbf{Bid Simulators:} Παρέχουν έτοιμες καμπύλες (Google Ads, Bing) για την $x_t$ και την $p_t$.
        \item \textbf{Προσέγγιση μέσω Regression:} Σε περίπτωση έλλειψης simulator:
        \begin{itemize}
            \item \textbf{Logistic Regression:} Για την εκτίμηση της $x_t$ (CTR).
            \item \textbf{Linear Regression:} Για την εκτίμηση της πληρωμής $p_t$.
        \end{itemize}
        \item \textbf{Ανθεκτικότητα:} Ο WIN-EXP παραμένει αποτελεσματικός ακόμα και με θορυβώδεις εκτιμήσεις.
    \end{itemize}
    
\end{frame}

\section{Πειράματα και Σύνοψη}
\begin{frame}{Πειραματική Αξιολόγηση}
    \begin{columns}[t]
        \begin{column}{0.55\textwidth}
            \textbf{Setup:} Weighted GSP, 20 πλειοδότες, 5.000 γύροι.
            \begin{itemize}
                \item \textbf{Αποτελέσματα:}
                    \begin{itemize}
                        \item Ο WIN-EXP υπερέχει σταθερά του EXP3 σε όλες τις κατηγορίες αντιπάλων.
                        \item Η απόδοση παραμένει σταθερή ακόμα και με πολύ πυκνή διακριτοποίηση (πολλά bins).
                        \item Υψηλή ανθεκτικότητα σε σενάρια με θορυβώδη δεδομένα (noisy data).
                    \end{itemize}
            \end{itemize}
        \end{column}
        
        \begin{column}{0.45\textwidth}
            \centering
            \begin{figure}
                \includegraphics[width=\textwidth]{images/b_comparison_winexp_adversaries.png}
                \caption*{\tiny Σύγκριση σωρευτικής μεταμέλειας (WIN-EXP vs Adversaries)}
            \end{figure}
        \end{column}
    \end{columns}
\end{frame}

\subsection{Σύνοψη}
\begin{frame}{Σύνοψη}
    \begin{itemize}
        \item \textbf{Βασικό Εύρημα:} Η άγνωστη αξία στις δημοπρασίες αποτελεί μια καλοήθη (benign) μορφή ελλιπούς πληροφόρησης.
        \item \textbf{Αποδοτικότητα:} Επίτευξη ρυθμών μεταμέλειας που προσεγγίζουν το σενάριο πλήρους πληροφόρησης (full information).
        \item \textbf{Lipschitz και Διακριτοποίηση:} Η Lipschitz συνέχεια αποτελεί τη μαθηματική θεμελίωση για τη χρήση πλέγματος σημείων (bins), επιτρέποντας υψηλή ανάλυση προσφορών χωρίς υποβάθμιση του ρυθμού σύγκλισης.
        \item \textbf{Πρακτική Αξία:} Άμεση εφαρμογή μέσω υπαρχόντων εργαλείων (Bid Simulators) ή στατιστικής εκτίμησης (Regression).
        \item \textbf{Μελλοντική Έρευνα:} Επέκταση σε περιβάλλοντα με δυναμικά μεταβαλλόμενο ανταγωνισμό ή καθυστερημένη ανατροφοδότηση.
    \end{itemize}
\end{frame}

\section{Βιβλιογραφία}
\begin{frame}[allowframebreaks]{Βιβλιογραφικές Αναφορές}
    \bibliographystyle{plain}
    \bibliography{references}
\end{frame}

\section{Τέλος}
\begin{frame}[plain,c]
    \begin{center}
        \Huge Τέλος\\
        \vspace{0.8cm}
        
        \large Ευχαριστώ για την προσοχή σας!\\
        \vspace{0.5cm}
        
        \qrcode[height=3cm]{https://github.com/gpetrousov/mab_assignment_demokritos}
        
        \vspace{0.3cm}
        \small \href{https://github.com/gpetrousov/mab_assignment_demokritos}{github.com/gpetrousov/mab\_assignment\_demokritos}
    \end{center}
\end{frame}

\end{document}